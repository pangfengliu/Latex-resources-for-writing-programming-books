\documentclass[11pt,twoside]{book}

\usepackage{CJK}
\usepackage{programmingbook}

\begin{document}

\begin{CJK}{UTF8}{bsmi}
  
\title{Latex Resources for Writing Programming Books}
\author{劉邦鋒 \\ 臺灣大學資訊工程學系}

\begin{titlepage}
\maketitle
\end{titlepage}

\chapter{Examples}

{\tt programmingbook} is a package that I use to write programming books.
It requires listing package and you can use it to write books in Chinese (with {\tt CJK}).
The latex source will look like this.

\begin{verbatim}
\documentclass[11pt,twoside]{book}

\usepackage{CJK}
\usepackage{programmingbook}

\begin{document}
\begin{CJK}{UTF8}{bsmi}
% your book here.
\end{CJK}
\end{document}
\end{verbatim}

\section{Program Listing}

Use {\tt programlisting} to list a program.
The first argument is the file name, and the second argument is the caption.

\begin{verbatim}
\programlisting{scan-print.c}{從鍵盤讀入再將其顯示}
\end{verbatim}

The previous code will produce the following.

\programlisting{scan-print.c}{從鍵盤讀入再將其顯示}

\section{Input/Output Listing}

Use {\tt programinput} and {\tt programoutput} to list program input
and output.
Note that we use {\tt file.in} and {\tt file.out} as the default file names to print if given the name {\tt file}.

\begin{verbatim}
\programinput{scan-print}
\programoutput{scan-print}
\end{verbatim}

The previous code will produce the following.

\programinput{scan-print}
\programoutput{scan-print}

\end{CJK}
\end{document}
